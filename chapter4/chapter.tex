\ifx\allfiles\undefined
\documentclass[12pt, a4paper, oneside, UTF8]{ctexbook}
\def\path{../config}
\usepackage{amsmath}
\usepackage{amsthm}
\usepackage{amssymb}
\usepackage{graphicx}
\usepackage{mathrsfs}
\usepackage{mathtools}
\usepackage{pifont}  % 多种符号
\usepackage{upgreek}  % 正体希腊字母
\usepackage{enumitem}
\usepackage{geometry}
\usepackage[colorlinks, linkcolor=black]{hyperref}
\usepackage{stackengine}
\usepackage{yhmath}
\usepackage{extarrows}
\usepackage{unicode-math}
\usepackage{caption}
\usepackage{ulem}  % 各种标记线
\usepackage{xargs}  % 多缺省命令
\usepackage{bm}  %  加粗数学字符 

\usepackage{fancyhdr}
\usepackage[dvipsnames, svgnames]{xcolor}
\usepackage{listings}
\usepackage{zhlipsum}  % 随机中文文本, 测试用
\usepackage{lipsum}  % 随机英文文本, 测试用

\definecolor{mygreen}{rgb}{0,0.6,0}
\definecolor{mygray}{rgb}{0.5,0.5,0.5}
\definecolor{mymauve}{rgb}{0.58,0,0.82}

\graphicspath{{figure/}, {../figure/}, {config/}, {../config/}}

% 设置图标签格式
\captionsetup[figure]{
    labelfont={bf},labelformat={default},labelsep=period,name={Fig.}}

\linespread{1.6}

\geometry{
    top=25.4mm, 
    bottom=25.4mm, 
    left=20mm, 
    right=20mm, 
    headheight=2.17cm, 
    headsep=4mm, 
    footskip=12mm
}

\setenumerate[1]{itemsep=5pt, partopsep=0pt, parsep=\parskip, topsep=5pt}
\setitemize[1]{itemsep=5pt, partopsep=0pt, parsep=\parskip, topsep=5pt}
\setdescription{itemsep=5pt, partopsep=0pt, parsep=\parskip, topsep=5pt}

\lstset{
    language=Mathematica,
    basicstyle=\tt,
    breaklines=true,
    keywordstyle=\bfseries\color{NavyBlue}, 
    emphstyle=\bfseries\color{Rhodamine},
    commentstyle=\itshape\color{black!50!white}, 
    stringstyle=\bfseries\color{PineGreen!90!black},
    columns=flexible,
    numbers=left,
    numberstyle=\footnotesize,
    frame=tb,
    breakatwhitespace=false,
} 

\numberwithin{equation}{section}  % 公式按节编号
\usepackage{tcolorbox}

\tcbuselibrary{most}

% 定义单独编号,其他四个共用一个编号计数 (原版),这里只列举了五种,其他可类似定义(未定义的使用原来的也可)
\newtcbtheorem[number within=section]{defn}
    {定义}{colback=Salmon!20, colframe=Salmon!90!Black, fonttitle=\bfseries}{def}

\newtcbtheorem[number within=section]{lemma}
    {引理}{colback=OliveGreen!10, colframe=Green!70, fonttitle=\bfseries}{lem}

% 使用另一个计数器可加参数 use counter from=lemma
\newtcbtheorem[number within=section]{them}
    {定理}{colback=SeaGreen!10!CornflowerBlue!10, 
           colframe=RoyalPurple!55!Aquamarine!100!,
           fonttitle=\bfseries}{them}

\newtcbtheorem[number within=section]{criterion}
    {准则}{colback=green!5, colframe=green!35!black, fonttitle=\bfseries}{cri}

\newtcbtheorem[number within=section]{corollary}
    {推论}{colback=Emerald!10, colframe=cyan!40!black, fonttitle=\bfseries}{cor}

\newtcbtheorem[number within=section]{proposition}
    {命题}{colback=red!5,colframe=red!75!black,fonttitle=\bfseries}{prop}
% red!5,colframe=red!75!black 警告框
% 使用格式是\begin{***}{}{} \end{***} ,需要两个 {}{} ,可以不填,但要有.
% 第一个 {} 填入别名 第二个为引用的 label 
% 引用方法为 \ref{def:xxx}

\newtheorem{example}{\indent \color{SeaGreen}{例}}[section]
\theoremstyle{plain}
\newtheorem*{rmk}{\indent 注}
\renewenvironment{proof}{\indent\textcolor{SkyBlue}{\textbf{证明:}}\;}{\qed\par}
\newenvironment{solution}{\indent\textcolor{SkyBlue}{\textbf{解:}}\;}{\qed\par}
% \def\d{\mathrm{d}}
\def\Rs{\mathbb{R}}
\def\Cs{\mathbb{C}}
\def\Zs{\mathbb{Z}}
\newcommand\mpi{\uppi}
\newcommand\mi{\mathrm{i}} 
\newcommand\me{\mathrm{e}}
\newcommand*{\dif}{\mathop{}\!\mathrm{d}}  % 微分算符 d
% \newcommand{\bs}[1]{\boldsymbol{#1}}
\newcommand{\ora}[1]{\overrightarrow{#1}}
\newcommand{\myspace}[1]{\par\vspace{#1\baselineskip}}
\newcommand{\xrowht}[2][0]{\addstackgap[.5\dimexpr#2\relax]{\vphantom{#1}}}
\newenvironment{ca}[1][1]{\linespread{#1} \selectfont \begin{cases}}{\end{cases}}
\newenvironment{vx}[1][1]{\linespread{#1} \selectfont \begin{vmatrix}}{\end{vmatrix}}
\newcommand{\tabincell}[2]{\begin{tabular}{@{}#1@{}}#2\end{tabular}}
\newcommand{\pll}{\kern 0.56em/\kern -0.8em /\kern 0.56em}
\newcommand{\bit}[1]{\symbfit{#1}}  % 加粗斜体
\newcommand{\Arg}[1]{\mathrm{Arg}\;#1}  % 辐角 Arg
\newcommand{\Div}[1]{\mathrm{div}\;#1}  % 散度 div
\newcommand{\grad}[1]{\mathrm{grad}\;#1}  % 梯度 grad
\newcommand{\Rot}[1]{\mathrm{rot}\;#1}  % 旋度 rot
\newcommand{\re}[1]{\mathrm{Re}\;#1}  % 实部 Re
\newcommand{\im}[1]{\mathrm{Im}\;#1}  % 虚部 Im
\newcommand*{\colorstar}[1][black]{\textcolor{#1}{\ast\quad}}  % 星星开头的重点行 (可改变星星颜色)
% 彩色的文字及下划线
\newcommandx*{\coloruline}[3][1=black, 2=black, usedefault]{
    \textcolor{#1}{\!\uline{\textcolor{#2}{#3}}}}  % 未完成此功能,下划线换行问题

\def\myIndex{0}  % 封面
% \input{\path/cover_package_\myIndex.tex}

\def\myTitle{复变函数}
\def\myAuthor{高峰}
\def\myDateCover{2023 年 6 月 13 日}
\def\myDateForeword{\today}
\def\myForeword{前言}
\def\myForewordText{
    
}
\def\mySubheading{}


\begin{document}
% \input{../config/cover}
\else
\fi

\chapter{留数定理}

\section{留数定理}

\[a_{-1} = \res f(z_0),\]
\textcolor{blue}{$a_{-1}$ 为 $f(z)$ 在 $z_0$ 的留数 (或残数)。}

\begin{defn}{留数定理}{}
    设函数 $f(z)$ 在回路 $l$ 所围区域 $B$ 上除有限个孤立奇点 $b_1$, $b_2$, $\cdots$, $b_n$ 
    外解析,在闭区域 $\overline{B}$ 上除  $b_1$, $b_2$, $\cdots$, $b_n$ 外连续,则:
    \[\oint_l f(z) \dif z = 2 \mpi \mi \sum_{j=1}^{n} \res f(b_{j}).\]
\end{defn}
\noindent \colorstar $l$ 以外无 $\forall$ 有限远奇点时,
$\ointclockwise_l f(z) \dif z = - 2 \mpi \mi a_{-1} = 2 \mpi \mi \res f(-\infty)$,
其中 $\res f(-\infty) = - a_{-1}(\infty)$。\\
\noindent \ding{172} 若 $z_0$ 是 $f(z)$ 的单极点,$\lim_{z\to z_0} ((z-z_0)f(z)) = \res f(z_0)$。\\
\noindent \ding{173} 若 $f(z) = \frac{P(z)}{Q(z)}$,其中 $P(z)$,$Q(z)$ 在 $z_0$ 解析,
$z_0$ 是 $Q(z)$ 的一阶零点,$P(z_0)\neq 0$,则 $z_0$ 为 $f(z)$ 的单极点,且 $\res f(z_0) = \frac{P(z_0)}{Q'(z_0)}$。\\
\ding{174} 若 $z_0$ 是 $f(z)$ 的 $m$ 阶极点,
\[\lim_{z\to z_0} ((z-z_0)^m f(z)) = a_{-m}\]
\[\res f(z_0) = \lim_{z\to z_0} \frac{1}{(m-1)!} (\frac{\dif^{m-1}}{\dif z^{m-1}}
((z-z_0)^m f(z))).\] 
\ding{175} 若 $z_0$ 是 $f(z)$ 的本性奇点,$\res f(z_0) = a_{-1}(z_0)$。(只能展开为洛朗级数求)\\
\textcolor{blue}{这 4 项均可作为极点判据和系数求法。}

\noindent \colorstar[blue] 使用留数定理求解路积分的步骤:\\
\ding{172} 找到所有奇点,判断奇点是否在 $l$ 内\\
\ding{173} 判断 $l$ 内的奇点的类别\\
\ding{174} 计算各奇点的留数\\
\ding{175} 应用留数定理得到积分的值。

\noindent \colorstar[blue] $f(z)$ \textcolor{blue}{(只有孤立奇点)} 在全平面上所有各点的留数之和为零。\\
\colorstar[blue] 求有些极限复杂的情况,可以先乘 $(z-z_0)^m$,再使用洛必达法则。


\section{应用留数定理计算实变函数定积分}

\noindent \textbf{类型一}
\[\int_0^{2\mpi} R(\cos x, \sin x) \dif x\]
令 $z = e^{\mi x}$,有 $\cos x = \frac{1}{2} (z + z^{-1})$,
$\sin x = \frac{1}{2\mi} (z - z^{-1})$,$\dif x = \frac{1}{\mi z} \dif z$。\\
原积分转化为 
$I = \oint_{\left\lvert z\right\rvert} R(\frac{z+z^{-1}}{2}, \frac{z-z^{-1}}{2\mi})
\frac{\dif z}{\mi z}$。

\noindent \textbf{类型二}
\[\int_{-\infty}^{\infty} f(x) \dif x,\]
\ding{172}  $f(z)$ 在实轴上无奇点;
\ding{173}  $f(z)$ 在上半平面除有限个奇点外是解析的;
\ding{174}  当 $z$ 在上半平面及实轴上 $\to\infty$ 时,$zf(z)$ 一致地 $\to 0$。

若 $f(x) = \frac{\varphi(x)}{\psi(x)}$,上述条件意味着 $\psi(x)$ 没有实的零点,
$\psi(x)$ 的次数至少高于 $\varphi(x)$ 两次。

\noindent 主值\quad $\mathscr{T} \int_{-\infty}^{\infty} f(x) \dif x 
= \lim_{R\to\infty} \int_{-R}^{R} f(x) \dif x$
% TODO: \mathscr{T}
\[\int_{-\infty}^{\infty} f(x) \dif x = 2\mpi \mi \{f(z)\mbox{\ 在上半平面所有奇点的留数之和}\}\]

\noindent \textbf{类型三}
\[\int_{0}^{\infty} F(x) \cos mx \dif x,\]
\[\int_{0}^{\infty} G(x) \sin mx \dif x,\]
并满足:\ding{172} 积分区间是 $[0, +\infty]$;\\
\ding{173} 偶函数 $F(z)$和奇函数 $G(z)$ 在实轴上没有奇点,
在上半平面除有限个奇点外是解析的;\\
\ding{174} 当 $z$ 在上半平面或实轴上 $\to\infty$ 时,$F(z)$ 和 $G(z)$ 一致地 $\to 0$。
\[\int_{0}^{\infty} F(x) \cos mx \dif x = 
\mpi \mi \{ F(z) e^{\mi m z} \mbox{\ 在上半平面所有奇点的留数之和}\},\]
\[\int_{0}^{\infty} G(x) \sin mx \dif x = 
\mpi \{ G(z) e^{\mi m z} \mbox{\ 在上半平面所有奇点的留数之和}\}.\]
\noindent \colorstar 约当引理

\noindent 实轴上有单极点的情形:\\
(1) $f(x)$ 在实轴上有单极点 $z=\alpha$,$f(z)$ 满足类型二或类型三的条件, 则:
\[\int_{-\infty}^{\infty} f(x) \dif x = 2\mpi \mi \sum_{\mbox{上半平面}} \res f(z) + \mpi \mi \res f(\alpha).\]
(2) 若实轴上有有限个单极点,为:
\[\int_{-\infty}^{\infty} f(x) \dif x = 2\mpi \mi \sum_{\mbox{上半平面}} \res f(z) + \mpi \mi \sum_{\mbox{实轴上}}\res f(z).\]

\noindent \colorstar $\int_{0}^{\infty} \frac{\sin mx}{x} \dif x = \frac{\mpi}{2}\quad (m>0)$,
$\int_{0}^{\infty} \frac{\sin mx}{x} \dif x = - \frac{\mpi}{2}\quad (m<0)$


\noindent \textbf{类型三公式的推广}
\begin{align*}
    \int_{-\infty}^{\infty} f(x) 
    \left\{ 
    \begin{lgathered} 
        \cos mx\\
        \sin mx
    \end{lgathered}   
    \right\}
    \dif x,\quad (m>0)
\end{align*}
当 $f(z)$ 在实轴无奇点,在上半平面除了有限个孤立奇点 $z_k$,$(k=1,2,\cdots,n)$ 之外处处解析,
且 $f(z)$ 在包含实轴的上半平面,
当 $z\to\infty$ 时, $f(z)$ 一致趋于 $0$,则:
\[\int_{-\infty}^{\infty} f(x) e^{\mi m x} \dif x 
= 2 \mpi \mi \sum_{k=1}^n \res (f(z_k)\cdot e^{\mi m z_k}),\]
其中,$\res (f(z_k)\cdot e^{\mi m z_k})$ 是 $f(z)\cdot e^{\mi m z}$ 的留数。\\
\textcolor{blue}{(无奇偶函数的限制了)}\\
\textcolor{red}{实部$=$实部,虚部$=$虚部}

\noindent \textbf{类型四}\ 含对数函数的积分
\[\int_{0}^{\infty} f(x) \ln x \dif x.\]

\ifx\allfiles\undefined
\end{document}
\fi