\usepackage{tcolorbox}

\tcbuselibrary{most}

% 定义单独编号,其他四个共用一个编号计数 (原版),这里只列举了五种,其他可类似定义(未定义的使用原来的也可)
\newtcbtheorem[number within=section]{defn}
    {定义}{colback=Salmon!20, colframe=Salmon!90!Black, fonttitle=\bfseries}{def}

\newtcbtheorem[number within=section]{lemma}
    {引理}{colback=OliveGreen!10, colframe=Green!70, fonttitle=\bfseries}{lem}

% 使用另一个计数器可加参数 use counter from=lemma
\newtcbtheorem[number within=section]{them}
    {定理}{colback=SeaGreen!10!CornflowerBlue!10, 
           colframe=RoyalPurple!55!Aquamarine!100!,
           fonttitle=\bfseries}{them}

\newtcbtheorem[number within=section]{criterion}
    {准则}{colback=green!5, colframe=green!35!black, fonttitle=\bfseries}{cri}

\newtcbtheorem[number within=section]{corollary}
    {推论}{colback=Emerald!10, colframe=cyan!40!black, fonttitle=\bfseries}{cor}

\newtcbtheorem[number within=section]{proposition}
    {命题}{colback=red!5,colframe=red!75!black,fonttitle=\bfseries}{prop}
% red!5,colframe=red!75!black 警告框
% 使用格式是\begin{***}{}{} \end{***} ,需要两个 {}{} ,可以不填,但要有.
% 第一个 {} 填入别名 第二个为引用的 label 
% 引用方法为 \ref{def:xxx}

\newtheorem{example}{\indent \color{SeaGreen}{例}}[section]
\theoremstyle{plain}
\newtheorem*{rmk}{\indent 注}
\renewenvironment{proof}{\indent\textcolor{SkyBlue}{\textbf{证明:}}\;}{\qed\par}
\newenvironment{solution}{\indent\textcolor{SkyBlue}{\textbf{解:}}\;}{\qed\par}